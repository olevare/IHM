\documentclass[a4paper,11pt]{report}

\usepackage[french]{babel}
\usepackage[T1]{fontenc}
\usepackage[utf8]{inputenc}

\begin{document}

\title{\textbf{Distribution Neutre de Sujets}\\Scénarios d'utilisation}
\author{Tristan Cossin, Morgan Fouque, Guilhem Trauchessec}


\begin{document}
\maketitle

\section*{Scénario 1: Etudiant Lambda}
Soit Jean-Charles un étudiant. Il est demandé à Jean-Charles de trouver un groupe de travail pour une matière en particulier mais, malheureusement pour lui, il ne connait pas beaucoup de monde. On lui propose alors l'outil de Distribution Neutre de Sujets afin de résoudre ce problème.

Il se rend à l'adresse indiquée et commence à utiliser l'application. Jean-Charles se rend compte que des critères sont mis en place : impossible de noter tout le monde 1/9 ou 9/9, le forcant à noter ses camarades et les sujets avec rigueur.

Lorsqu'il a noté à convenance Sujets et Etudiants (Plus de rouge lors de la validation), l'application propose à Jean-Charles une première assignation. Il a alors le choix d'accepter ou de refuser la solution qui lui est proposé. D'autres lui seront proposées.

Il jugera de la pertinance des assignations grâce au code couleur des différents éléments.

\section*{Scénario 2: Etudiant Discret}
Soit Millie une étudiante. Millie est dans un groupe de connaissances depuis la L1, mais à chaque devoir de groupe, c'est elle qui fait la quasi-intégralité du travail demandé. 

L'origine du soucis lui vient du fait qu'elle n'arrive pas à s'imposer dans les décisions, et que lorsqu'on lui propose de travailler ensemble, elle ne peut pas décliner sciemment.

Ainsi, l'application lui permet de faire ses choix sans dépendre des autres, et si elle connait des gens qui délègue le travail à d'autres, elle pourra faire en sorte de ne pas tomber avec eux.

C'est aussi l'occasion pour elle d'élargir son cercle de connaissances.

\end{document}